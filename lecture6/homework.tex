\documentclass[11pt]{article}
\usepackage{hyperref}
\usepackage{enumitem}


\title{The Economics of Cybersecurity: Homework 5}
% \date{Due: February 20 at {\bf 5:00pm}}
\author{}

\begin{document}

\maketitle


\section*{}

Last week, we read one empirical paper and one experimental paper. Both papers produced data that better inform us on the state of the world. Critiquing these works generally amounts to critiquing their methodologies or critiquing their interpretations of findings. 

Next week, we are going to discuss a different class of paper, which we may classify as ``position papers''.
Position papers generally try to frame some problem from a particular perspective and then defend their chosen position via evidence and arguments. 
Position papers---which are commonplace in fields like policy, law, and economics---are not necessarily strict scientific inquiry per se but still play an important role in the scientific process by introducing and defending novel paradigms (and potentially urging scientific communities towards a certain research agenda). 
% Cybersecurity (being a discipline that encompasses multiple fields of study) has a history of position papers that are 

Our next readings are three position papers that advocate for a particular ``doctrine'' of cybersecurity. The meaning of ``doctrine'' here comes from the legal concept of a law doctrine, which can be defined as a widely followed guiding rule or set of rules for resolving legal disputes (usually established over time through common law precedent).
To illustrate, some examples of well-known legal doctrines in the United States are:
\begin{itemize}
    \item {\bf Rescue doctrine:} If your actions puts someone in danger, you are liable not only for any harms caused to that person but also liable for harms caused to anyone attempting to rescue that person as well. 
    \item {\bf Fair use doctrine:} Copyright materials may be used in limited ways without permission from the copyright holder for purposes such as quotations, scholarly criticism, and news reporting. 
    \item {\bf Castle doctrine:} Individuals have the right to use reasonable force (including up to deadly force) against home intruders (precise definitions of ``reasonable'' may vary between legal jurisdictions).
    \item {\bf Fairness doctrine:} Broadcast media operating under a broadcast license must present controversial issues in a way that fairly represents differing viewpoints (abolished by the FCC in 1987)
\end{itemize}

Note that these doctrines do not aim to solve every edge case or address every nuance, but instead aim to provide a high-level guideline or set of guidelines for resolving disputes. 

The same can be done for cybersecurity.
Read the following papers in the following order: 

\begin{enumerate}
    \item ``\href{https://direct.mit.edu/daed/article/140/4/70/26918/Doctrine-for-Cybersecurity}{Doctrine for Cybersecurity}'' (2011), by Mulligan \& Schneider
    \item ``\href{https://academic.oup.com/cybersecurity/article/3/3/173/3836936?searchresult=1}{Coercion in cybersecurity: What public health models reveal}'' (2017), by Weber
    \item ``\href{https://dl.acm.org/doi/abs/10.1145/3411504.3421217}{WaC: A New Doctrine for Hardware Security}'' (2020), by Hastings \& Sethumadhavan
\end{enumerate}


\section*{Dialectical Paper Discussion (Presentation \#2)\\ (10 points --- In Class Presentation score)}

Without firm methodologies or data, what can we as readers do to convince ourselves that a proposed position in a position paper is the right one? 
How might we determine if we should accept or reject a proposed doctrine of security?
Unfortunately there seems to be no clear one-size-fits-all algorithm for answering this question.
One method we \textit{can} use, though, is the dialectical approach---using reasoned argumentation to (hopefully) arrive at truth.
\\
\\
Your goal is to do the following:

\begin{enumerate}
    \item Choose one of the three papers above
    \item Choose a a position---either in \textit{favor} of the paper's position, or \textit{against} it. Write down your chosen paper and position \href{https://docs.google.com/spreadsheets/d/1SdaMpeCo4CE8o0U_irGhWmVXJYHybRWoiX4uwZYMhCE/edit?usp=sharing}{here} (make sure to select the sheet for Class \#6).
    \item Come to class prepared to defend your chosen position against your chosen paper. 
\end{enumerate}


\noindent The format of each paper's discussion will be as follows:
\begin{enumerate}
    \item Affirmative Construction (6 minutes) --- The Affirmative (the person arguing in favor of a position) presents an outline of the paper, the main arguments therein, and their reasoning for why the class should accept the paper's position. 
    \item Cross Examination (3 minutes) --- The Negative (the person arguing against a position) asks clarifying questions about the Affirmative's exposition of the paper and tries to find or expose any flaws in the Affirmative's arguments. 
    \item Negative Construction (4 minutes) --- The Negative (the person arguing against a position) presents the case for why we should not accept the paper's position
    \item Cross Examination (3 minutes) --- The Affirmative asks clarifying questions about the Negative's position and tries to find or expose any flaws in the Negative's arguments
    \item Group discussion (8 minutes) --- The class discusses the paper and may ask the Affirmative and the Negative questions on the content of the paper or clarification questions on their arguments. 
    \item Resolution (2 minutes) --- The Affirmative and the Negative find common ground between their positions. 
\end{enumerate}

The goal is not for this to be a debate with a winner and loser but instead to see how well the papers hold up under argumentation between an advocate and an antagonist.
In other words, if you are the Affirmative, the Negative is not your enemy!
They too are trying to establish the quality of the paper in question, but from an opposing perspective.

You may have an easier time arguing a position if you agree with your chosen position, but please note that this is not actually a requirement!
For example, if you choose to defend a given paper but do not actually agree with its position, consider yourself to be akin to a public defender who knows their defendant is guilty---everyone (or paper) is entitled to a fair defense regardless of the merits of any individual case. 
Likewise, if you are arguing against a paper but personally agree with it, consider your role to be to play devil's advocate.



\end{document}
