\documentclass[11pt]{article}
\usepackage{geometry}
\usepackage{graphicx}
\usepackage{placeins}
\usepackage{multirow}
\usepackage{float}
\usepackage{amsmath}
\usepackage[normalem]{ulem}
\usepackage[table,xcdraw]{xcolor}

\graphicspath{{figures/}}

\setlength{\parskip}{\baselineskip}%
\setlength{\parindent}{0pt}%
\geometry{left=2.5cm,right=2.5cm,top=2.5cm,bottom=2.5cm}

\title{The Economics of Cybersecurity --- Lecture 6 Notes}
\date{February 20, 2024}

\author{Adam Hastings}


\begin{document}
\maketitle

\section*{Project Topics}

\textit{Have each group briefly describe their class project}

\section*{Doctrine for Cybersecurity}

Three prior doctrines proposed:
\begin{itemize}
    \item Prevention
    \item Risk Management 
    \item Deterrence through Accountability
\end{itemize}

Each with flaws though. What are they?

``Cybersecurity is non-rivalrous and non-excludable'' is the justification for the public goods model. {\it Is this correct though? Since it underpins much of the argument.}

{\it How is (cyber)vaccination a ``tragedy of the commons''?}

{\it Does ``herd immunity'' really apply to security these days?}

Public health can serve as a guide. Not always a 1-to-1 mapping but can inspire methods:
\begin{itemize}
    \item Education of professionals/Certification---{\it How do you do this in a domain that is constantly changing? Attackers are often one step ahead. Kind of dismissed by the industry}
    \item Law, e.g. introducing liability---{\it What are the different types?}
    \item Standards---{\it What about security theatre?}
\end{itemize}


\subsection{Pros}
\begin{itemize}
    \item Introduces idea of doctrine as a guiding principle
\end{itemize}

\subsection{Cons}
\begin{itemize}
    \item Too much coercion
    \item Too much loss of privacy
    \item Security not a public good in many regards
    \item This paper was pre-Snowdon. Public opinion on government surveillance is different now. 
    \item Patching --- sysadmins might not even know about all the systems on their network or how to patch
    \item Info sharing with government orgs---what do private sector companies get in return? Nothing? Are they supposed to report out of goodness?
\end{itemize}


\section*{Coercion in cybersecurity: What public health models reveal}

Interesting to read in light of COVID, which started 3 years after this paper was published.


\subsection{Summary}

\begin{itemize}
    \item Review: Doctrines are high-level conceptual frameworks. E.g. MAD during cold war---had goals (deterrence) means (second strike capability), and desired outcomes (prevention of war)
    \item Doctrines have issues though because unlike in nuclear war, where there is a shared desired outcome (no war), in cybersecurity, there are different parties with meaningfully different goals. E.g. is protection of free speech a cybersecurity outcome? Do we want to still be able to attack other countries? No agreement on means. 
    \item Public health has been proposed as an analogous doctrine for security by many.
    \item Three points of contention to this view: 1) Security not exactly a public good, 2) Heavily relies on government intervention; many questions about what appropriate government interventions are 3) May involve significant coercion. 
    \begin{enumerate}
        \item Is security really a public good at all?
        \begin{itemize}
            \item Gives 2x2 matrix of goods which we discussed in first week of class
            \item Nature determines rivalrousness but not excludability, which is determined by policy choices. Security (and health) are maybe better described as club goods (excludable, non-rival) via e.g. vaccine passes and quarantines.
        \end{itemize}
        \item Public health model glosses over the key role of government in providing public goods. 
        \begin{itemize}
            \item Public goods are often underprovisioned in the marketplace (free riding, tragedies of the commons).
            \item Governments are often seen as the ones to ensure proper production of public goods. This can be done via education, monitoring, or coercion. 
            \item E.g. mandatory reporting 
            \item What level of coercion should be exerted? Not clear.
            \item Economic models exist, but rely on variables that are hard or impossible to measure in practice. 
        \end{itemize}
        \item Neglects how important coercion has been in major public health achievements
        \begin{itemize}
            \item Examines how disease control, automobile safety, smoking, and obesity have been addressed through various levels of coercion.
            \item Health offices have requires mandatory reporting (of e.g. tuberculosis), mandatory vaccination, surveillance, enforced quarantines
            \item Justified by literal saving of lives. Very quantifiable. Might not be the case in cybersecurity!
            \item Levels of coercion in e.g. automobile safety (seatbelts, BAC limits) are already a restrained balance. But were eventually accepted because less coercive measures (e.g. public education) failed to have adequate impact. 
        \end{itemize}
    \end{enumerate}
    \item To summarize: public health has been proposed as a doctrine for security, but Weber claims these deteriorate under scrutiny. If we want to follow these metaphors, it's going to require a greater coercive authority.
\end{itemize}

\subsection{Pros}
\begin{itemize}
    \item Adds to the discourse. Challenges common assumptions.
\end{itemize}

\subsection{Cons}
\begin{itemize}
    \item Rambling! Doesn't forcefully state the point. Needs to summarize sections or something
    \item No data to back up points (but fun citations---Rosseau 1762!)
\end{itemize}

\section*{A New Doctrine for Hardware Security}



\end{document}
