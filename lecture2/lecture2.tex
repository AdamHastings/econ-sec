\documentclass[11pt]{article}
\usepackage{geometry}
\usepackage{graphicx}
\graphicspath{{figures/}}

\setlength{\parskip}{\baselineskip}%
\setlength{\parindent}{0pt}%
\geometry{left=2.5cm,right=2.5cm,top=2.5cm,bottom=2.5cm}

\title{The Economics of Cybersecurity --- Lecture 2 Notes}
\date{January 23, 2024}

\author{Adam Hastings}


\begin{document}


\maketitle

\section*{Pre-Class}
\begin{itemize}
    \item Write title, course number, hours, on blackboard
    \item Write out sections of discussion
\end{itemize}

\section{Homework Recap (6:10)} 

(Discuss agenda for the day. Write agenda on board)

\subsection{Systems diagrams review (6:10)}

(Depending on class size, pick on 2--3 to share their chosen problem and diagram)

Conclusion: It's kind of messy! Good segue...

\subsection{``Wicked problem'' paper review (6:25)}

Discussion:
\begin{itemize}
    \item Initial thoughts?
    \item Applicable to security? Which points? Discuss
    \item What are our thoughts on this type of ``science''? There's no hypothesis being tested here. Sort of like ``here's something we intuited and came up with a name for it''. Is it science? How do we trust it? It may be the best we can do. 
    \item Very different from other types of CS papers you may have read before.
\end{itemize}

This concludes our discussion of ``systems-thinking'' although the mindset is going to pervade much of the rest of the semester.

\section{Foundations of Economics (6:30)}

So far we've talked about security as a system and maybe talked about some open issues in security but up to this point we haven't really talked about security as an {\it economics} problem.

\subsection{The Supply Curve (6:30)}

\subsection{The Demand Curve (6:40)}

\subsection{Pareto efficiency (6:50)}?

\section{Market Failures in Security (7:00)}

{\it Ask: What kinds of assumptions are needed to make markets efficient?}


This section is really the ``why'' this subject needs to exist. 
Markets are wonderful and they solve many problems of the allocation of goods and services.
But let's review some of the required assumptions needed for efficient markets.

{\it Ask: Which of the above requirements might be violated in the world of security?}

Some suggested answers:
\begin{enumerate}
    \item Perfect information
\end{enumerate}

\section{Paper Discussion: Why Information Security is Hard (7:00)}

Introductory discussion things to note:
\begin{enumerate}
    \item This is a seminal paper. So where are the experiments? Where are the graphs and tables? There are none. How can this be? (If you're looking for a technical rigor to this class, we will make sure we cover that too).
    \item A big part of this class is going to be studying methods. This paper has no methods! But in future papers we will highlight methods used. 
\end{enumerate}

\subsection{Types of Market Failures in Security}

\subsection{Conclusion}

“In general, where the party who is in a position to protect a system is not the party who would suffer the results of security failure, then problems may be expected.”



\end{document}
