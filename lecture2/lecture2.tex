\documentclass[11pt]{article}
\usepackage{geometry}
\usepackage{graphicx}
\usepackage[table,xcdraw]{xcolor}

\graphicspath{{figures/}}

\setlength{\parskip}{\baselineskip}%
\setlength{\parindent}{0pt}%
\geometry{left=2.5cm,right=2.5cm,top=2.5cm,bottom=2.5cm}

\title{The Economics of Cybersecurity --- Lecture 2 Notes}
\date{January 23, 2024}

\author{Adam Hastings}


\begin{document}


\maketitle

\section*{Pre-Class}
\begin{itemize}
    \item Write title, course number, hours, on blackboard
    \item Write out sections of discussion
\end{itemize}

\section{Homework Recap (6:10)} 

(Discuss agenda for the day. Write agenda on board)

\subsection{Systems diagrams review (6:10)}

(Depending on class size, pick on 2--3 to share their chosen problem and diagram)

Conclusion: It's kind of messy! Good segue...

\subsection{``Wicked problem'' paper review (6:25)}

Don't want to spend too much time on this so let's briefly review.

Discussion:
\begin{itemize}
    \item Initial thoughts?
    \item Applicable to security? Which points? Discuss
    \item What are our thoughts on this type of ``science''? There's no hypothesis being tested here. Sort of like ``here's something we intuited and came up with a name for it''. Is it science? How do we trust it? It may be the best we can do. 
    \item Very different from other types of CS papers you may have read before.
\end{itemize}

This concludes our discussion of ``systems-thinking'' although the mindset is going to pervade much of the rest of the semester.

\section{Big Ideas in Economics (6:30)}

So far we've talked about security as a system and maybe talked about some open issues in security but up to this point we haven't really talked about security as an {\it economics} problem.

This class doesn't assume any economics knowledge. So we're going to establish a basic foundation.
You may have seen some of this stuff before if you've ever taken an economics class before.
But we need to 

\subsection{Big Idea \#1: Goods}

Economics is fundamentally about the distribution of goods and services. 

{\it Ask: What is a good?}

A {\bf good} is an item that provides value or utility for someone. Something that someone wants. Something that someone is willing to sacrifice something to obtain. Example: A table, or a barrel of oil.

{\it Ask: What are some goods we deal with in security?} Hardware, obviously. Software can also be a good (even though it is sort of intangible).

{\it Ask: What about services? How are they different?}

A {\bf service} is an act that someone performs because someone else values the act and is willing to pay for it. 
A good is transferrable. A service is not. 
Example: a haircut.

Different from goods in that they are always {\bf intangible} (whereas goods can be tangible or intangible). 
Another difference is that services are {\bf non-transferrable}: Once a service is performed, it can't be transferred to someone else. 
If I can't a haircut I can't undo it and give it someone else.
Different from a transferrable good like a chair.

{\it Ask: What are some services we deal with in security?} Penetration testing, incident response. 

\subsubsection{Types of Goods}

We can taxonomize goods in a few ways.
Two main ones are used: rivalrousness and exclusivity.

(Write out 2x2 grid)

{\bf Rivalrous}: Consumption by one person $\rightarrow$ cannot be consumed by another.
Example: If I eat an apple, you can't also eat it.  

{\bf Exclusive}: Consumption can be restricted to certain people only.


\begin{table}[]
\begin{tabular}{lccll}
\cline{1-3}
\multicolumn{1}{|l|}{\cellcolor[HTML]{EFEFEF}}                       & \multicolumn{1}{c|}{\cellcolor[HTML]{EFEFEF}\textbf{Excludable}} & \multicolumn{1}{c|}{\cellcolor[HTML]{EFEFEF}\textbf{Non-Excludable}} &  &  \\ \cline{1-3}
\multicolumn{1}{|c|}{\cellcolor[HTML]{EFEFEF}\textbf{Rivarlrous}}    & \multicolumn{1}{c|}{private goods}                               & \multicolumn{1}{c|}{common-pool resources}                           &  &  \\ \cline{1-3}
\multicolumn{1}{|c|}{\cellcolor[HTML]{EFEFEF}\textbf{Non-Rivalrous}} & \multicolumn{1}{c|}{club goods}                                  & \multicolumn{1}{c|}{public goods}                                    &  &  \\ \cline{1-3}
                                                                        & \multicolumn{1}{l}{}                                             & \multicolumn{1}{l}{}                                                 &  & 
\end{tabular}
\end{table}


\subsection{Big Idea \#2: Some goods are more valuable to some people than to others}.



Some goods can even be more or less valuable to the same people depending on circumstances.
E.g. if it's a sunny day, I might not care to have an umbrella; if it's raining really hard, I'm really going to want one. 

\subsection{The Supply Curve (6:30)}

\subsection{The Demand Curve (6:40)}

\subsection{Pareto efficiency (6:50)}?

\section{Market Failures in Security (7:00)}

{\it Ask: What kinds of assumptions are needed to make markets efficient?}


This section is really the ``why'' this subject needs to exist. 
Markets are wonderful and they solve many problems of the allocation of goods and services.
But let's review some of the required assumptions needed for efficient markets.

{\it Ask: Which of the above requirements might be violated in the world of security?}

Some suggested answers:
\begin{enumerate}
    \item Perfect information
\end{enumerate}

\section{Paper Discussion: Why Information Security is Hard (7:00)}

Introductory discussion things to note:
\begin{enumerate}
    \item This is a seminal paper. So where are the experiments? Where are the graphs and tables? There are none. How can this be? (If you're looking for a technical rigor to this class, we will make sure we cover that too).
    \item A big part of this class is going to be studying methods. This paper has no methods! But in future papers we will highlight methods used. 
\end{enumerate}

\subsection{Types of Market Failures in Security}

\subsection{Conclusion}

“In general, where the party who is in a position to protect a system is not the party who would suffer the results of security failure, then problems may be expected.”



\end{document}
