\documentclass[11pt]{article}
\usepackage{geometry}
\usepackage{graphicx}
\usepackage[table,xcdraw]{xcolor}

\graphicspath{{figures/}}

\setlength{\parskip}{\baselineskip}%
\setlength{\parindent}{0pt}%
\geometry{left=2.5cm,right=2.5cm,top=2.5cm,bottom=2.5cm}

\title{The Economics of Cybersecurity --- Lecture 2 Notes}
\date{January 23, 2024}

\author{Adam Hastings}


\begin{document}


\maketitle

\section*{Pre-Class}
\begin{itemize}
    \item Write title, course number, hours, on blackboard
    \item Write out sections of discussion
\end{itemize}

\section{Homework Recap (6:10)} 

(Discuss agenda for the day. Write agenda on board)

\subsection{Systems diagrams review (6:10)}

(Depending on class size, pick on 2--3 to share their chosen problem and diagram)

Conclusion: It's kind of messy! Good segue...

\subsection{``Wicked problem'' paper review (6:25)}

Don't want to spend too much time on this so let's briefly review.

Discussion:
\begin{itemize}
    \item Initial thoughts?
    \item Applicable to security? Which points? Discuss
    \item What are our thoughts on this type of ``science''? There's no hypothesis being tested here. Sort of like ``here's something we intuited and came up with a name for it''. Is it science? How do we trust it? It may be the best we can do. 
    \item Very different from other types of CS papers you may have read before.
\end{itemize}

This concludes our discussion of ``systems-thinking'' although the mindset is going to pervade much of the rest of the semester.

\section{Big Ideas in Economics (6:30)}

So far we've talked about security as a system and maybe talked about some open issues in security but up to this point we haven't really talked about security as an {\it economics} problem.

This class doesn't assume any economics knowledge. So we're going to establish a basic foundation.
You may have seen some of this stuff before if you've ever taken an economics class before.
But we need to 

\subsection{Big Idea \#1: Goods}

Economics is fundamentally about the distribution of goods and services. 

{\it Ask: What is a good?}

A {\bf good} is an item that provides value or utility for someone. Something that someone wants. Something that someone is willing to sacrifice something to obtain. Example: A table, or a barrel of oil.

{\it Ask: What are some goods we deal with in security?} Hardware, obviously. Software can also be a good (even though it is sort of intangible).

{\it Ask: What about services? How are they different?}

A {\bf service} is an act that someone performs because someone else values the act and is willing to pay for it. 
A good is transferrable. A service is not. 
Example: a haircut.

Different from goods in that they are always {\bf intangible} (whereas goods can be tangible or intangible). 
Another difference is that services are {\bf non-transferrable}: Once a service is performed, it can't be transferred to someone else. 
If I can't a haircut I can't undo it and give it someone else.
Different from a transferrable good like a chair.

{\it Ask: What are some services we deal with in security?} Penetration testing, incident response. 

\subsubsection{Types of Goods}

We can taxonomize goods in a few ways.
Two main ones are used: rivalrousness and exclusivity.

(Write out 2x2 grid)

{\bf Rivalrous}: Consumption by one person $\rightarrow$ cannot be consumed by another.
Example: If I eat an apple, you can't also eat it.  

{\bf Excludable}: Consumption can be restricted to certain people only. 
Example: Concerts. You physically cannot get access without buying a ticket.

{\it Ask: What's an example of a non-excludable good?} Air. A lighthouse (everyone can take advantage of it).

I'm writing these as binary categories but in reality they exist more on spectrums.
Example: this class! Supposed to be available to those who pay tuition. But if someone wanted to audit, I wouldn't physically prevent them from entering the classroom (could even enhance if they contribute to the discussion!). So this class is semi-excludable.


\begin{table}[]
    \centering
    \begin{tabular}{lccll}
    \cline{1-3}
    \multicolumn{1}{|l|}{\cellcolor[HTML]{EFEFEF}}                       & \multicolumn{1}{c|}{\cellcolor[HTML]{EFEFEF}\textbf{Excludable}} & \multicolumn{1}{c|}{\cellcolor[HTML]{EFEFEF}\textbf{Non-Excludable}} &  &  \\ \cline{1-3}
    \multicolumn{1}{|c|}{\cellcolor[HTML]{EFEFEF}\textbf{Rivarlrous}}    & \multicolumn{1}{c|}{private goods}                               & \multicolumn{1}{c|}{common-pool resources}                           &  &  \\ \cline{1-3}
    \multicolumn{1}{|c|}{\cellcolor[HTML]{EFEFEF}\textbf{Non-Rivalrous}} & \multicolumn{1}{c|}{club goods}                                  & \multicolumn{1}{c|}{public goods}                                    &  &  \\ \cline{1-3}
                                                                            & \multicolumn{1}{l}{}                                             & \multicolumn{1}{l}{}                                                 &  & 
    \end{tabular}
\end{table}

\begin{enumerate}
    \item Private goods: rivalrous + excludable. Examples: GPUs, firewalls. As opposed to...
    \item Public goods: non-rivalrous, non-excludable. Examples: Air. Cybersecurity itself?
    \item Club goods: excludable, non-rivalrous. Examples: Antivirus software? Since it is intangible. 
    \item Common-pool resources: rivalrous but non-excludable. Example: Fish in the sea. Internet traffic?
\end{enumerate}


\subsection{Big Idea \#2: We can model the value of goods}

Some goods are more valuable to some people than to others.

Some goods can even be more or less valuable to the same people depending on circumstances.
E.g. if it's a sunny day, I might not care to have an umbrella; if it's raining really hard, I'm really going to want one. 

Acquiring goods usually involves trading off something you want for something else you want more. 

There are a few common methods that economists to capture how valuable things are to people.

\subsubsection{Indifference curves}

A typical way of expressing how much people value something is via indifference curves. This is a way of expressing how much someone values something in terms of something else. 

\begin{figure}[h]
    \centering
    \includegraphics*[width=2.5in]{indifference.png}
\end{figure}
 

\begin{itemize}
    \item X: the good in question
    \item Y: could be another good. But typically is a composite of all other goods!
\end{itemize}

This is a 2D space of possible goods you can acquire. We're going to make some simplifying assumptions and say that we're dealing with ``normal'' goods (like e.g. more is better. May not be a realistic assumption! Economics can handle more advanced cases but take an econ class if you want to learn more about that).

Each point is a different combination of goods, like $Q_x$ of good X and $Q_y$ of good Y. (Draw $(Q_x, Q_y)$ on the board) We call this point a {\bf bundle}.

An {\bf indifference curve} is the line connecting all the bundles that someone finds to be equally attractive (draw a few indifference curves).

{\it Ask: I drew it convex. Why?} Diminishing marginal utility. Lets use a computer systems analogy. Let's say I have a system where X is my CPU clock frequency (it's something we want! It's a good!) and Y is all other system design constraints. If my goal is to make a fast system, and my clock frequency is very low, this could be the bottleneck in system performance

Normal goods are usually convex like this. There are of course some exceptions.

{\it Ask: What does it mean if I draw the indifference curve as a straight line?} It means that X and Y are perfect substitutes---I don't care if I have one or the other. 

Another variation is indifference curves with a ``bliss point'' i.e. an optimum. single point w/ surrounding lines. Looks like a topographic map!

One important element of an indifference curve is that the slope at each point is equal to the {\bf marginal rate of substitution (MRS)}, which is the rate at which the consumer is wiling to substitute good $X$ for good $Y$.

\subsubsection{The budget line}

Recall that we usually deal with normal goods, so more = better. In this case people would always want to maximize $(q_x, q_y)$, i.e. the top-rightmost possible point. But people don't have unlimited budgets so we have to make some constraints.

A common assumption in many econ problems is that people are working with a fixed budget of money they have to spend. We can call this amount $m$. Then it necessarily follows that
$$p_Xq_X + p_Yq_Y = m$$
This is just the formula of a straight line {\it (draw negative sloped line)}.


\subsubsection{Composite goods}
One thing about this arrangement (indifference curves) is that it expresses how much someone values one good in terms of another. This might be useful if there were only two things people want. But if we want to know how much someone values apples, does this mean we need to make a new set of indifference curves for every possible other good out there? Apples vs pears? Apples vs corn? Apples vs iron ore? Apples vs movie theater tickets? No, there's a better solution. We can instead just let the other variable $Y$ be a {\bf composite good}, which represents ``everything else the consumer might want to consume'' (Varian). In this case we can just write $p_Y = 1$ since the price of one dollar is one dollar.

\subsubsection{Optimal Choice}

The {\bf optimal choice} is the point where the the budget line is tanget to the indifference curve. This is the most preferrable bundle of goods. We can call this bundle $(q_x^*, q_y^*)$ {\it (draw dashed lines connecting to tangent point $(q_x^*, q_y^*)$)}

At this point the marginal rate of substitution is equal to the slope of the budget line.

\subsubsection{Changing prices}

As prices change, the budget line and indifference curves intersect at different points {\it (draw \ref{fig:demand}-A)}.
As the price decreases, the budget line pivots outwards.


\subsection{The Demand Curve}

Another important way of expressing peoples wants is through demand curves. 
A {\bf demand curve} is the optimum quantity of a good as a function of its price. 
For reasons I'm not entirely clear on this is typically drawn with the {\it price on the Y axis even though demand is a function of price!}
(I think this might be because when economists talk about supply, it's the opposite where they view price as a function of demand, so putting both on the same graph means one gets stuck with the unconventional plotting.)



\begin{figure}[h]
    \centering
    \includegraphics*[width=3.5in]{demandcurve.png}
    \caption{Source: Intermediate Microeconomics by Hal Varian, 8th ed.}
    \label{fig:demand}
\end{figure}

\subsection{The Supply Curve}

Supply side economics is equally important---to economists. We won't focus on it so much in this class. In classical microeconomics it suffices to say that those who produce goods are incentivized to produce more when prices are higher. The amount that acutally gets produced is the point where the supply curves and demand curves intersect. This is called the {\bf market clearing price}. Of course this is a huge simplification and there are lots of caveats so go read an econ book if you want to learn more. 

\section{Market Failures in Security (7:00)}

{\it Ask: What kinds of assumptions are needed to make markets efficient?}


This section is really the ``why'' this subject needs to exist. 
Markets are wonderful and they solve many problems of the allocation of goods and services.
But let's review some of the required assumptions needed for efficient markets.

{\it Ask: Which of the above requirements might be violated in the world of security?}

Some suggested answers:
\begin{enumerate}
    \item Perfect information
\end{enumerate}

\section{Paper Discussion: Why Information Security is Hard (7:00)}

Introductory discussion things to note:
\begin{enumerate}
    \item This is a seminal paper. So where are the experiments? Where are the graphs and tables? There are none. How can this be? (If you're looking for a technical rigor to this class, we will make sure we cover that too).
    \item A big part of this class is going to be studying methods. This paper has no methods! But in future papers we will highlight methods used. 
\end{enumerate}

\subsection{Types of Market Failures in Security}

\subsection{Conclusion}

“In general, where the party who is in a position to protect a system is not the party who would suffer the results of security failure, then problems may be expected.”



\end{document}
