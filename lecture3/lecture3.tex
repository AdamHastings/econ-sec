\documentclass[11pt]{article}
\usepackage{geometry}
\usepackage{graphicx}
\usepackage{placeins}
\usepackage{multirow}
\usepackage{float}
\usepackage{amsmath}
\usepackage[normalem]{ulem}
\usepackage[table,xcdraw]{xcolor}

\graphicspath{{figures/}}

\setlength{\parskip}{\baselineskip}%
\setlength{\parindent}{0pt}%
\geometry{left=2.5cm,right=2.5cm,top=2.5cm,bottom=2.5cm}

\title{The Economics of Cybersecurity --- Lecture 2 Notes}
\date{January 23, 2024}

\author{Adam Hastings}


\begin{document}


\maketitle

\section*{Pre-Class}
\begin{itemize}
    \item Write title, course number, hours, on blackboard
    \item Write out sections of discussion
\end{itemize}


\section{The Market for Lemons}

\subsection{Opening questions}
\begin{itemize}
    \item What did people think?
    \item It was ``straightforward'' microeconomics but it makes assumptions and assumes that the reader knows what these assumptions are. Did anyone struggle to follow any of the arguments being made or struggle to follow the math? I think the math is deceptively simple.
    \item Akerlof won the Nobel Memorial Prize in Economics for basically this paper alone (maybe a couple others, but on this same topic). Is that surprising to you? (Note: Shared it with Stiglitz---a Columbia professor!)
    \begin{itemize}
        \item Side note--The Nobel Prize in Economics is officially the Sveriges Riksbank Prize in Economic Sciences in Memory of Alfred Nobel. Not one of the original Nobel Prizes! Administered by a bank!
    \end{itemize}
\end{itemize}



\subsection{Lemons Model}

I want to cover this model in full detail because it's such a bite-sized example of how we can use mathmatical modeling to explain and illustrate things. It might feel like excruciating detail since you already read the paper but like I said it has a lot of hidden assumptions and by going through the model we can air out and examine what these assumptions are.


\subsubsection{Asymmetric Information}

First, there's the assumption that the market is going to settle on a fixed price $p$. 


We also have two groups. Group 1 has a utility function of 
$$U_1 = M + \sum_{i=1}^N x_i$$
while Group 2 has a utility funciton of 
$$U_1 = M + \sum_{i=1}^N \frac{3}{2}x_i$$
Recall that utility is a quantitative representation of goodness, and that more is better. 

{\it Which group values cars more?} Group 2.

{\it Which group has cars?} Group 1.

{\it What does this mean about the inital allocation of goods? (This was a homework quesiton.)} It means the initial allocation is Pareto inefficient. This means we can make a Pareto optimization. If Group 2 collectively bought all the cars at price $p = \mu$, 

{\it Why call this Group 1 and Group 2? Why not just call them ``buyers'' and ``sellers''?} I think this is because calling them ``buyers'' and ``sellers'' makes it seem like Group 1 wants to sell cars and Group 2 wants to buy cars. That is true but slightly misleading. Group 1 still values cars (as per their utility funciton), and according to their utility function they still would actually buy a car if it was a good deal. Calling Group 1 the ``sellers'' makes it sound like they're going to sell no matter the price, which simply isn't true. Framing things using a utility function allows the model remain simple and allows us to assume that Groups 1 and 2 are the same, just with different valuations of cars.

For example, both groups have demand functions! Let's take a look at them.

Group 1's demand function looks like this:
\begin{align*}
D_1(p) = Y_1/p & \qquad \qquad \mu/p > 1 \\
D_1(p) = 0 & \qquad \qquad \mu/p < 1
\end{align*}

What does $\mu/p > 1$ mean? I found it easier to rewrite it as $\mu > p$. Recall that quality and price are not the same units but are on the range of 0 to 2. So $\mu > p$ just means that average quality is higher than the price, and $\mu < p$ means that average quality is lower than the price. 

{\it What happens when the average quality is lower than the price?} Well this depends on Group 1's utility function. The scaling factor for each car in Group 1's utility function is 1. Hence for {\it Group~1}, $\mu/p > 1$ is a bad deal. In the case of a bad deal, what is the demand going to be? 0. Hence $D_1(p) = 0, \quad \mu/p < 1$ as above.

What about the other case, where $\mu/p > 1$? Akerlof introduces new variables, $Y_1$ and $Y_2$ to represent Group~1 and Group~2's income respetively, including the income that results from selling cars. 

So we know that $\mu/p > 1$ is a good deal for Group~1 members, and they're going to want cars at this price, so what will the demand be? Somewhere in the above assumptions made is that cars are good, more cars are better, and that the value of each additional car is equal to its quality. So when the price is a good deal, Group~1 wants as many cars as they can get with their $Y_1$ income. And the amount they can get is $Y_1/p$. Does that make sense?

That's how Akerlof constructs the demand curves for Group 1. How does he construct the supply curve? 

We're assuming linear utility. If the price $p=0$, there are going to be 0 cars sold. If the price is $p=2$, there are going to be $N$ cars sold. This is a straight line with slope $N/2$ ({\it Draw this on the board with price $p$ on x-axis}). Hence 
\begin{equation}
    S_1 = \frac{pN}{2} \quad p \leq 2
\end{equation}
Note that I think this is a typo! Akerlof writes ``$S_2$'' instead of ``$S_1$''! This was the extra credit question in the homework. Good job if you spotted this. If the price $p>2$, we can just assume that all $N$ cars will be sold. 

What is the interpretation of the above? It means that the supply is subject to the price. This is just mathematically encoding the idea that sellers will only sell if the market price is above their car's quality, and will hold onto the car if the market price is below their car's quality. 

There's another derived equation here, for average quality:
\begin{equation}
    \mu = p/2
\end{equation}

Where does this come from? Recall that that at any price $p$, the cars that are sold will be of quality $x_i \leq p$ (which is valid because price and quality are normalized to the same [0,2] scale). If we draw out the PDF of the quality of cars that will sell at price $p$, what is the average quality? $p/2$.

Now let's look at the supply and demand curves for Group 2. What is a ``good deal'' for Group 2? How much utility do they get from a car? They get $\frac{3}{2}x_i$. So on average they get $\frac{3}{2}\mu$ for each car. So they will buy when $\frac{3}{2}\mu > p$.
As with Group 1, there demand is subject to their own income $Y_2$ and the price of the cars $p$. 

\begin{align*}
D_1 = Y_2/p & \qquad \qquad \frac{3}{2}\mu > p \\
D_1 = 0 & \qquad \qquad \frac{3}{2}\mu < p
\end{align*}

And Group 2 has no cars so 

$$S_2 = 0$$

Now we can write the combined demand function $D(p, \mu) = D_1 + D_2$. This follows straightforwardly from the above two demand curves. 

If $p<\mu$, then both Group 1 and Group 2 want cars, so $D(p, \mu) = Y_1/p + Y_2/p = (Y_1 + Y_2)/p$.

If $\mu < p < \frac{3\mu}{2}$, then only Group 2 wants cars, so $D(p, \mu) = Y_2/p$.

If $ p > \frac{3\mu}{2}$, then $D(p, \mu) = 0$. 

This is the full model in the case of the asymmetric information, where Group 1 only sells cars if the market price is beneath their car's value. 

But price is $p$ while average quality is $\mu = p/2$. Of the three demand curve cases above, which one does this correspond to? It corresponds to $D = 0, p>3\mu/4 = p > \frac{3}{4}p$. So demand is zero. No sales take place despite the fact that there are Group 1 members who have cars they are willing to sell at prices Group 2 members are willing to pay.

In other words, in this model, there are Pareto optimizations that could occur but the market does not produce this outcome. This system remains Pareto inefficient. This is why this situation is called a market failure.

\subsubsection{Symmetric Information}

Akerlof also shows what happens when there is symmetric information. What changes? 

\begin{align*}
    S(p) = N & \qquad p > 1\\
    S(p) = 0 & \qquad p < 1
\end{align*}

It's a step function. Group 1 will sell when the price is greater than 1.

The demand curves are similar:
\begin{align*}
D(p) &= (Y_1 + Y_2)/p & \qquad p < 1\\
D(p) &= (Y_2)/p & \qquad 1 < p < 3/2\\
D(p) &= 0 & \qquad p > 3/2\\
\end{align*}

In classic microeconomics, the price of a good is the point at which the supply curve and the demand curve intersect. When does $S(p)=D(p)$?

Let's break this down by cases. If $p < 1$, then supply is 0 and demand is $(Y_1 + Y_2)/p$, which will be greater than 0 if we assume that these three variables are greater than 0. There will be no intersection of supply and demand in this range.

Likewise, if $p > 3/2$, demand is 0 but supply is $N$. Again we assume $N>0$ so there will be no intersection of supply and demand here. 

That means that the intersection of supply and demand will be between $p=1$ and $p=3/2$. What are the conditions needed for $p$ to be in this price range?

There are three possible cases. 

Case 1 is that $N > Y_2/p$:

\begin{figure}[h]
    \centering
    \includegraphics*[width=4.5in]{equilibrium1.png}
    \label{fig:equ1}
\end{figure}
\FloatBarrier


In this case, supply and demand meet when $p=1$. And hence $N > Y_2/p$,
\begin{equation}
    p=1 \quad \text{if} \quad Y_2 < N
\end{equation}


\newpage
Case 2 is when $N = Y_2/p$:

\begin{figure}[h]
    \centering
    \includegraphics*[width=4.5in]{equilibrium2.png}
    \label{fig:equ2}
\end{figure}
\FloatBarrier


This case requires that $1 < p < 3/2$. So if $N =  Y_2/p \implies p = Y_2/N$, then we get 
\begin{align*}
    1 < \frac{Y_2}{N} < \frac{3}{2}  \\
    1 > \frac{N}{Y_2} > \frac{2}{3}
\end{align*}
\begin{equation}
    \frac{2Y_2}{3} < N < Y_2
\end{equation}

\newpage
Case 3 is when $N < Y_2/p$:
\begin{figure}[h]
    \centering
    \includegraphics*[width=4.5in]{equilibrium3.png}
    \label{fig:equ3}
\end{figure}
\FloatBarrier

This means that $p=\frac{3}{2}$. Plugging into the above and rearranging yields:
\begin{equation}
    N < \frac{2Y_2}{3}
\end{equation}


\subsection{Closing questions}

This is a great example of how to show a point using a mathematical model. A few quesitons come up:
\begin{enumerate}
    \item Do you think Akerlof created the model first and then found out that it had this interesting property? (I highly doubt it. More likely that he had the intuition, and then created the model to support the intuition).
    \begin{itemize}
        \item What do we think about creating models to ``mathify'' an intuition we have? Is this just adding math as we see fit to make our arguments seem more impressive and more bulletproof? 
    \end{itemize}
    \item What did we think about the example applications that Akerlof gives? (Insurance, hiring practices, cost of dishonesty, credit markets in underdeveloped economies)
    \item What did we think about the proposed solutions? (brand repuation/chains, seller guarantees, credentials/licensing) (I thought they were pretty basic and obvious. That's fine though---not his job to fix in this paper!)
    \item This paper was rejected twice before publication because the reviewers pointed out that if the model was correct, then no used cars would ever be sold. In other words, the real-world evidence that used cars do in fact sell means that this model is incorrect.
    \begin{itemize}
        \item What do we think of this complaint?
        \item Does this mean that one of the assumptions that Akerlof makes is wrong? Which one?
        \item A surprising takeaway might be that your model doesn't even need to accurately reflect the real world to be useful. What are your thoughts on that?
    \end{itemize}
    \item Do you find it surprising how nonchalant Akerlof is about this? He doesn't seem to be writing about it as if it's a Nobel Memorial Prize-winning paper.
\end{enumerate}

\section{The Economics of Information Security Investment}

The other paper we read was in many ways very similar. It made a mathematical model of something, and then demonstrated how this model exhibits interesting behavior.

\subsection{Opening discussion}

\begin{itemize}
    \item Focuses on the risk of securing a ``dataset''. But seems to be a much more general model than that? Seems to cover any cases where you have something worth protecting.
\end{itemize}

\subsection{Model description}

Three parameters:
\begin{itemize}
    \item $\lambda$: The loss given that a breach occurs. Finite, less than some very large number $M$.
    \item $t$: Threat. Probability of a threat occurring. $t \in [0,1]$. Pinned at a fixed value $t > 0$ (since no amount of investment is going to change $t$).
    \item $v$: Vulnerability. Probability that a threat is successful. $v \in [0,1]$. Hence $v=0 \implies$ perfect security, and $v=1 \implies$ zero security i.e. public information.
\end{itemize}

Expected loss is therefore $\lambda t v$. The potential loss $L = t \lambda$.

Assumption: Investment can reduce the the vulnerability. Denoted by $z>0$, in same units (i.e. dollars) as $\lambda$. 
Let $S(z,v)$ denote the probability that information set with vulnerability $v$ will be breached. 

Then the authors make three assumptions: 
\begin{itemize}
    \item A1: $S(z,0) = 0$ for all $z$. {\it Ask: Can someone interpret what this means?} (It means that completely secure data remains perfectly secure regardless of the investment)
    \item A2: $\forall v, S(0,v) = v$.  {\it Ask: Can someone interpret what this means?} (It means that if there is no investment, the probability of a breach is just equal to its vulnerability score.
    \item A3: $\forall v \in (0,1), \forall z, S_z(z,v) < 0 \text{ and } S_{zz}(z,v)> 0$, where $S_z$ denotes the partial derivative w.r.t. $z$ and $S_{zz}$ denotes the partial derivative of $S_z$ w.r.t. $z$. Interpretation: there are diminishing marginal returns to investment (example: $f(x) = 1/x$). Also assume that as $\forall v \in (0,1,), z \rightarrow \infty \implies \textit{lim } S(z,v) \rightarrow 0$, i.e. the probability of a breach can be arbitrarily close to 0 with sufficient investment. 
\end{itemize}

What are our thoughts on these assumptions?

Next the authors assume that firms are risk neutral. {\it What does this mean?} (It was defined in a footnote). It means that you are indifferent towards risk as long as the expected value of interest remains constant. E.g. you would just as likely accept \$10 as you would a 50/50 chance between \$20 and \$10.

Firms are going to make the investment that maximizes their personal benefit. To do this, the authors define an expected benefit of investment in information security, EBIS. Recall that the original expected loss was $v t \lambda = v L$. So with investment the new loss is defined as 
$$ \text{EBIS}(z) = [v - S(z,v)]L$$

This is how much you expect to lose with investment $z$. But this doesn't take into account the actual investment itself. What firms really want to minimize is expected \textit{net} benefit of investment in information security, or E\textit{N}BIS, which is 
$$\text{ENBIS}(z) \qquad = \qquad \text{EBIS}(z) - z \qquad=\qquad [v - S(z,v)]L - z$$
What firms want to do is find the $z^*(v)$ that minimizes ENBIS.

This is a concave fucntion (upside-down cup) starting at 0 and reaching some maximum before decreasing back to 0. 

The derivative with respect to $z$ is then 
$$ ENBIS_z(z^*) = S_z(z^*,v)L - 1 = 0$$
$$ \implies \quad -S(z^*,v)L = 1$$

\begin{figure}[h]
    \centering
    \includegraphics*[width=4.5in]{fig1.png}
    \label{fig:fig1}
\end{figure}
\FloatBarrier


Initial thoughts:
\begin{itemize}
    \item Wise to let $\lambda$ cover all possible types of losses. Keeps the model simple. 
    \item I think the model's biggest limitation is that  you need to know the function $S(z,v)$. Very difficult (maybe impossible!) to do so. (but a few classes of candidate $S(z,v)$ functions are provided!)
    \item Economic modeling like this has a huge similarity with machine learning. What is it? (In both cases, the challenge is usually defining a loss function, and then achieving optimum points either by direct analysis or by computation).
    \item That brings up another point----this is a ``one round'' (aka ``one shot'') game (unlike in ML). 
    \begin{itemize}
        \item What do we think of this? Is this realistic?
        \item Security is an ongoing process. Attackers always responding to Defenders' behavior...
        \item But at the same time, much easier to mathematically work with one shot games than to deal with differential equations.
        \item What are the alternatives? Giant formulae? Really painful derivations? Simulation perhaps...? Many differential equations have closed form solutions.
        \item A quote in a footnote: ``A model is supposed to reveal the essence of what is going on: your model should be reduced to just those pieces that are required to make it work'' (Varian). Thoughts?
    \end{itemize}
\end{itemize}

\subsection{Model Exploration}

One thing this paper has that The Market for Lemons paper does not is model exploration. 

In Section II of the paper (where the model is defined), the shape of $S(z,v)$ is left pretty undefined (some constraints on the sign of its first and second derivatives though).

\subsubsection{Candidate $S(z,v)$ 1}

Candidate 1 is $S^I = \frac{v}{(\alpha z + 1)^\beta}$, with $\alpha > 0, \beta \geq 1$. Should we check that this satisfies the above conditions?

\begin{figure}[h]
    \centering
    \includegraphics*[width=4.5in]{class1.png}
    \label{fig:class1}
    \caption{Class I candidate function for $S(z,v)$ (with $z$ along the x-axis)}
\end{figure}
\FloatBarrier

\begin{figure}[h]
    \centering
    \includegraphics*[width=4.5in]{fig2.png}
    \label{fig:class1}
    \caption{Class I: Expected Loss as a function of vulnerability $v$. Higher investments (higher levels of $z$) produce less expected loss.}
\end{figure}
\FloatBarrier

Since we now have a definition of $S(z,v)$, we can comptue the derivative and actually find the optimal security investment, which in this case is 
$$z^{I*}(v) = \frac{(v \beta \alpha )^{1/(\beta + 1)} - 1}{\alpha}$$ 
which I will not derive here and leave as an exercise for the student (it might be in the appendix).

\subsubsection{Candidate $S(z,v)$ 2}


\subsection{Concluding discussion}

Questions:
\begin{itemize}
    \item How practically useful is this? Do you think that firms are using this model to determine how much to invest in cybersecurity?
    \begin{itemize}
        \item I don't think I've ever heard of anyone using this in the real world\dots
        \item Yet this paper has almost 2000 citations. What does that mean?
        \item Why? Is the model too abstract or too simple to accurately capture the nature of information security? Or is it too hard to come up with good estimates of values like $t$ and $\lambda$?in real world situations? Is it something else?
    \end{itemize}
\end{itemize}



\end{document}
