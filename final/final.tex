\documentclass[11pt]{article}
\usepackage{amsmath}
\usepackage{amssymb}
\usepackage{geometry}
\usepackage{graphicx}
\usepackage{placeins}
\usepackage{multirow}
\usepackage{float}
\usepackage{amsmath}
\usepackage[normalem]{ulem}
\usepackage[table,xcdraw]{xcolor}
\usepackage{datetime}
\usepackage{hyperref}
\settimeformat{hhmmsstime}

\graphicspath{{figures/}}

\setlength{\parskip}{\baselineskip}%
\setlength{\parindent}{0pt}%
\geometry{left=2.5cm,right=2.5cm,top=2.5cm,bottom=2.5cm}

\title{The Economics of Cybersecurity --- Final}

\date{{\bf Time:} Tuesday May 7, 7:10pm -- 10:00pm \\ {\bf Location:} 516 Hamilton}
% \author{Adam Hastings}


\begin{document}
\maketitle
% \begin{center}
% {\bf Location:} 516 Hamilton

% {\bf Time:} Tue May 7, 7:10pm -- 10:00pm
% \end{center}


\subsection*{Paper (80 points)}

Please write a 10-page paper on your class project. 
\begin{itemize}
    \item Use the ACM \texttt{sigconf} (i.e. conference) paper format. You can use the \texttt{nonacm} keyword in the \LaTeX \ \ \texttt{documentclass} declaration to suppress conference information \\ (e.g. use \texttt{\textbackslash documentclass[sigconf,nonacm]\{acmart\}}).
    \item If you want to compile \LaTeX \ \  locally, you can follow the instructions \href{https://www.acm.org/publications/proceedings-template}{here}. You can also compile via Overleaf, which has an ACM template \href{https://www.overleaf.com/latex/templates/association-for-computing-machinery-acm-sig-proceedings-template/bmvfhcdnxfty}{here}.
    \item Write your paper as if it were going to be submitted to a scientific conference, following the precedents set by the other papers we've read in class. 
    \item The paper will be due at the start of the final exam period. 
\end{itemize}

\subsection*{Presentation (20 points)}

Please also prepare a 12-minute presentation (+ 3 minutes Q\&A) on your work to present during the schedule final exam period. Make sure to motivate the problem you are trying to solve as well as the methods you used to solve them. Please consider incorporating feedback from previous presentations into your final presentation.

% \section{Grading}


% \begin{itemize}
%     \item Paper
%     \begin{itemize}
%         \item 20 points --- Overall Clarity
%     \end{itemize}
%     \item Presentation
%     \begin{itemize}
%         \item 5 points --- Overall clarity 
%         \item {\bf Note:} This will be your fourth presentation to the class. Please consider incorporating feedback from previous presentations into your final presentation.
%     \end{itemize}
% \end{itemize}



\end{document}
