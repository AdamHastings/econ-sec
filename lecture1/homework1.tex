\documentclass[11pt]{article}
\usepackage{hyperref}

\title{The Economics of Cybersecurity: Homework 1}
\date{Due: January 23 at 5:00pm}
\author{}

\begin{document}

\maketitle

\section{Paper 1 (5 points)}

We have not yet talked about approaches towards {\it solving} security problems using an economic mindset 

Read the paper ``\href{https://link.springer.com/article/10.1007/bf01405730}{Dilemmas in a general theory of planning}'' by Rittel \& Webber. Answer the following question:

\begin{enumerate}
    \item Rittel \& Webber list ten criteria for a problem to be considered a ``wicked problem''. Security may be considered to be such a problem. Which three criteria are the most applicable or relevant to security? Write one paragraph for each or $\sim$500 words total, defending your choices. Cite any sources used, including the use of generative AI tools.  
\end{enumerate}

\section{Paper 2 (5 points)}

Read the paper ``\href{https://ieeexplore.ieee.org/abstract/document/991552}{Why information security is hard - an economic perspective}''. Come prepared to class ready to discuss the paper.


\section{Systems Diagram (10 points)}

\begin{enumerate}
    \item Choose a security problem that you are interested in. Add your name and chosen topic to the spreadsheet \href{https://docs.google.com/spreadsheets/d/1SdaMpeCo4CE8o0U_irGhWmVXJYHybRWoiX4uwZYMhCE/edit#gid=0}{here}. You may draw from the provided list of suggested topics in the spreadsheet if you do not have a particular topic of interest. Please pick a topic that is different from the other students in the class.
    \item Create a system-level diagram of the problem as demonstrated in class. Be prepared to present your work at the beginning of next class.
\end{enumerate}

Notes:
\begin{itemize}
    \item Systems are not processes! A process can be thought of as a path {\it through} a system. The implication here is that there should be no element of ordering or time involved in your systems diagram.
\end{itemize}

\end{document}
