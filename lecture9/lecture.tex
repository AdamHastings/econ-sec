\documentclass[11pt]{article}
\usepackage{geometry}
\usepackage{graphicx}
\usepackage{placeins}
\usepackage{multirow}
\usepackage{float}
\usepackage{amsmath}
\usepackage[normalem]{ulem}
\usepackage[table,xcdraw]{xcolor}

\graphicspath{{figures/}}

\setlength{\parskip}{\baselineskip}%
\setlength{\parindent}{0pt}%
\geometry{left=2.5cm,right=2.5cm,top=2.5cm,bottom=2.5cm}

\title{The Economics of Cybersecurity --- Lecture 9 Notes}
\date{March 19, 2024}

\author{Adam Hastings}


\begin{document}
\maketitle



\section{Hack for Hire: Exploring the Emerging Market for Account Hijacking}


\begin{itemize}
    \item Let's revisit the key points from this paper. 
    \begin{itemize}
        \item What is a honeypot?
        \item Setup: Creating fake personas
        \begin{itemize}
            \item Victims were U.S.-based
            \item Always used Gmail-based email address 
            \item The authors created victims in the native language of the hacking service (``Natasha Belkin''). Why?
            \item How did the authors create the illusion that these were real email accounts? (Enron email corpus w/ changed dates \& names)
            \item How else did the authors create the illusion that these were real people? (Fake Facebook accounts, blogs, fictitious small business. Also fake associate personas as well!)
            \item Do we think this worked? (The majority didn't even attack! Outright scam, or did they smell something was fishy? If the latter, how would they have known?)
        \end{itemize}
        \item What were the key findings?
        \begin{itemize}
            \item 
        \end{itemize}
    \end{itemize}
    \item What do we think about the fact that email is more or less considered the root of trust for most online activities?
    \begin{itemize}
        \item Who benefits from this arrangement?
        \item What are the risks?
        \item What are the alternatives? (Hardware-backed tokens? What are the downsides here? The costs?)
        \item Is it surprising that email compromise is a significant target for attackers? Probably not. 
    \end{itemize}
    \item What is {\it targeted } attacking? 
    \begin{itemize}
        \item How is it different from untargeted attacking?
        \item How common is it?
        \item Which one is more profitable?
    \end{itemize}
    \item What are the economics of targeted attacking?
    \begin{itemize}
        \item 
    \end{itemize}
    \item What can we learn from this style of research?
    \begin{itemize}
        \item What are the ethical concerns here?
        \item Could this data have been collected any other way?
    \end{itemize}
    \item Let's talk about the economics elements of this work.
    \begin{itemize}
        \item The authors found that attackers want to double their pay if the account hijack requires a 2FA bypass. Can we use this as a proxy for the ``cost'' that 2FA imposes on an attacker?
        \item Why or why not?
        \item If so, is this a reasonable method of doing cost-benefit analysis of security defenses? Can we rank the efficiency of defenses based on the ratio of (cost) : (cost to compromise)?
        \item If so, can we do this for all types of defenses? Why or why not? 
        \item If so, who should pay for this type of research? Academia? Government? Industry? What if the attackers find out they've been honeypotted?
    \end{itemize}
    \item If the average cost to compromise is \$300, what are the implications?
    \begin{itemize}
        \item The economics of the cost of this service mean that there must be a significant reward for doing so.
        \item This means that either the financial reward must be big (maybe it's the email of a high-ranking business executive, and compromising their email can enable lucrative BEC scams)
        \item OR you are not motivated by money. Maybe you are a spy agency trying to gain intelligence on someone.
    \end{itemize}
    \item What can be done to reduce this type of crime?
    \begin{itemize}
        \item As mentioned earlier, increase the adoption of authenticator devices. 
    \end{itemize}
    \item What has changed since 2019, when this paper was published?
    \begin{itemize}
        \item Surprisingly little, it seems. I sort of remember SMS-2FA starting around ~2016 maybe. And this still seems to be the status quo, eight years later. 
        \item I wouldn't be surprised if these results replicated today. 
    \end{itemize}
\end{itemize}


\end{document}
