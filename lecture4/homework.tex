\documentclass[11pt]{article}
\usepackage{hyperref}

\title{The Economics of Cybersecurity: Homework 3}
\date{Due: February 6 at {\bf 5:00pm}}
\author{}

\begin{document}

\maketitle


\section*{Paper 1 (10 points --- to be applied to Class Presentations score)}

Our next class topic will be behavioral economics and security, which is the study of how people make decisions. A notable tenant of behavioral economics is that it assumes (and in many cases quantifies) how people are ``predictably irrational'', often in measurable and systemic ways. These insights have many applications to security.

Rather than cover every aspect of this large subdiscipline of economics, we will instead teach each other the principles of the domain by presenting security papers that employ behavioral economics. Read the survey paper \href{https://dl.acm.org/doi/abs/10.1145/3054926}{Nudges for Privacy and Security: Understanding and Assisting Users' Choices Online}. Pay close attention to Section 2 to get a sense of the questions that behavioral economics aims to understand. Then in Section 3, pay attention to how the principles from Section 2 are applied to answer questions of security. 
\\ 

\noindent Choose a security-related paper from one of these sections:

\begin{itemize}
    \item 3.2.2 Education in Privacy and Security
    \item 3.2.3 Feedback for Privacy and Security
    \item 3.3.4 Presentation Nudges for Privacy and Security
    \item 3.4.2 Defaults in Privacy and Security
    \item 3.5.2 Costs of Privacy and Security
    \item 3.5.3 Incentives for Security
\end{itemize}


You will give a mini-presentation on your chosen paper during the next class. Fill out your choice \href{https://docs.google.com/spreadsheets/d/1SdaMpeCo4CE8o0U_irGhWmVXJYHybRWoiX4uwZYMhCE/edit?usp=sharing}{here} (make sure you select the ``HW3'' tab at the bottom). Please choose a unique topic from your classmates. Prepare to present your chosen paper for 10 minutes, followed by 5 minutes of discussion and question-answering, for a total of 15 minutes of class time.

You are free to create your presentation as you see fit. However, you may find it useful to follow this template to help keep you on track:
\begin{itemize}
    \item {\bf Introduction} ({\it $<30$ seconds}): Your name, the paper's title, the authors' names, relevant background info on authors (e.g. any other notable works of theirs that is relevant to current paper).
    \item {\bf Background and context} ({\it 3 minutes}): What questions are the authors trying to answer in this paper? Why is this an important question to answer? For this particular assignment---where in the survey paper did you find your chosen paper? What are the principles of behavioral economics (as outlined in Section 2) that are being applied?
    \item {\bf Methods} ({\it 3 minutes}): What do the authors do to answer the questions posed in the Background section? Is this a theory paper? An experimental paper? An empirics paper? Are the methods chosen by the authors the appropriate ones to answer their questions? 
    \item {\bf Results} ({\it 2 minutes}): What did the authors find using their methods? You likely will not have time to present all data from the paper. Choose the most important results to report. Consider briefly summarizing what you do not cover in depth.
    \item {\bf Applications, impact, and discussion} ({\it 1 minute}): What are the implications of this work? Who is affected by the discoveries made in this paper?
    \item {\bf Conclusion} ({\it $<30$ seconds}): A quick summary of the last 9.5 minutes.
\end{itemize}

\noindent Please keep the following in mind when making your presentation:

\begin{itemize}
    \item Your audience will either read your slides or listen to you, but not both. Generally the less text you put on a slide, the better. 
    \item Do not put images on a slide unless they serve a purpose. Too often I see someone put an image in a slide (because they think it will catch the audience's attention or something, I assume) and then not actually discuss the image on the screen. This is a waste of the audience's attention, who will look at the image instead of listening to you, and not learn anything, because you did not discuss the image.
    \item Presenting a paper is not the same as defending it. You are allowed (and even encouraged!) to be critical of the paper you choose. If you think the methods used by the researchers seem questionable, or that the results do not follow from the data, please say so. 
    \item As part of the 5-minute discussion/Q\&A section following your presentation, you may be asked to answer questions but also may use this time to lead a discussion on the topic.
\end{itemize}

\noindent Please submit either a PDF of your slides as part of the your HW3 submission, or if using something like Google Slides, share the link with me (hastings@cs.columbia.edu) and submit the link via a comment in Courseworks. (This is so that I can make a combined slidedeck of everyone's slides and minimize the amount of time spent plugging in laptops and adjusting screen settings.)




\section*{IRB Training (10 points)}

Research in cybersecurity economics often requires experimentation involving human subjects. 

\begin{enumerate}
    \item Go to \href{rascal.columbia.edu}{rascal.columbia.edu}
    \item In the top navigation bar, to go ``Training Center''
    \item Click on ``Course Listings'' and search for Course \#TC0087 --- Human Subjects Protection Training. Click ``Assign to Myself''.
    \item Go back to ``Training Center''. Go to ``My Training To-Do List''. Course \#TC0087 should appear. Complete this training.
    \item After you have completed the training, go back to ``Training Center'' and click on ```View Certified Test History'''. Generate a certificate for Course \#TC0087. Submit this certificate as your assignment.
\end{enumerate}


\end{document}
